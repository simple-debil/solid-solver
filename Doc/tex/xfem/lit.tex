
\begin{thebibliography}{3}
\bibitem{Magadova2012}
Магадова Л. А., Силин М. А., Глущенко В. Н. Нефтепромысловая химия. Технологические аспекты и материалы для гидроразрыва пласта //издательский центр РГУ нефти и газа им. ИМ Губина учебное пособие.–2012.–423 с. – 2012.


\bibitem{GuptaDiss2016}
Gupta P. A generalized finite element method for the simulation of non-planar three-dimensional hydraulic fracture propagation : дис. – University of Illinois at Urbana-Champaign, 2016.
\bibitem{WeberDiss2016}
Weber N. The XFEM for Hydraulic Fracture Mechanics Die XFEM für die hydraulische Bruchmechanik.
\bibitem{Lecampion2018}
Lecampion B., Bunger A., Zhang X. Numerical methods for hydraulic fracture propagation: a review of recent trends //Journal of natural gas science and engineering. – 2018. – Т. 49. – С. 66-83.
\bibitem{Kolawole2020}
Kolawole O., Ispas I. Interaction between hydraulic fractures and natural fractures: current status and prospective directions //Journal of Petroleum Exploration and Production Technology. – 2020. – Т. 10. – №. 4. – С. 1613-1634.

\bibitem{PereiraDiss2010}
Pereira J. P. Generalized finite element methods for three-dimensional crack growth simulations : дис. – University of Illinois at Urbana-Champaign, 2010.
\bibitem{Belytschko2009}
Belytschko T., Gracie R., Ventura G. A review of extended/generalized finite element methods for material modeling //Modelling and Simulation in Materials Science and Engineering. – 2009. – Т. 17. – №. 4. – С. 043001.
\bibitem{Fries2010}
Fries T. P., Belytschko T. The extended/generalized finite element method: an overview of the method and its applications //International journal for numerical methods in engineering. – 2010. – Т. 84. – №. 3. – С. 253-304.
\bibitem{Karihaloo2011}
Karihaloo B., Xiao Q. Z. Accurate simulation of mixed-mode cohesive crack propagation in quasi-brittle structures using exact asymptotic fields in XFEM: an overview //Journal of Mechanics of Materials and Structures. – 2011. – Т. 6. – №. 1. – С. 267-276.
\bibitem{Fries2014}
Fries T. P. Overview and comparison of different variants of the XFEM //PAMM. – 2014. – Т. 14. – №. 1. – С. 27-30.
\bibitem{flemisch2016}
Flemisch B., Fumagalli A., Scotti A. A review of the XFEM-based approximation of flow in fractured porous media //Advances in Discretization Methods. – 2016. – С. 47-76.
\bibitem{Ahmadkanth2019}
Kanth S. A. et al. Elasto Plastic Crack Growth by XFEM: A Review //Materials Today: Proceedings. – 2019. – Т. 18. – С. 3472-3481.

\bibitem{Gupta2014}
Gupta P., Duarte C. A. Simulation of non‐planar three‐dimensional hydraulic fracture propagation //International Journal for Numerical and Analytical Methods in Geomechanics. – 2014. – Т. 38. – №. 13. – С. 1397-1430.
\bibitem{Zielonka2014}
Zielonka M. G. et al. Development and validation of fully-coupled hydraulic fracturing simulation capabilities //Proceedings of the SIMULIA community conference, SCC2014. – 2014. – С. 19-21.
\bibitem{Gupta2015}
Gupta P., Duarte C. A. Coupled formulation and algorithms for the simulation of non-planar three-dimensional hydraulic fractures using the generalized finite element method //International Journal for Numerical and Analytical Methods in Geomechanics. – 2016. – Т. 40. – №. 10. – С. 1402-1437.
\bibitem{Liu2016}
Liu F. et al. A stabilized extended finite element framework for hydraulic fracturing simulations //International Journal for Numerical and Analytical Methods in Geomechanics. – 2017. – Т. 41. – №. 5. – С. 654-681.
\bibitem{Weber2016}
Weber N. The XFEM for Hydraulic Fracture Mechanics Die XFEM für die hydraulische Bruchmechanik.
\bibitem{Haddad2016}
Haddad M., Sepehrnoori K. XFEM-based CZM for the simulation of 3D multiple-cluster hydraulic fracturing in quasi-brittle shale formations //Rock Mechanics and Rock Engineering. – 2016. – Т. 49. – №. 12. – С. 4731-4748.
\bibitem{Gupta2017}
Gupta P., Duarte C. A. Coupled hydromechanical‐fracture simulations of nonplanar three‐dimensional hydraulic fracture propagation //International Journal for Numerical and Analytical Methods in Geomechanics. – 2018. – Т. 42. – №. 1. – С. 143-180.
\bibitem{Liu2017}
Liu F., Gordon P. A., Valiveti D. M. Modeling competing hydraulic fracture propagation with the extended finite element method //Acta Geotechnica. – 2018. – Т. 13. – №. 2. – С. 243-265.
\bibitem{Luo2018}
Luo Z. et al. Seepage-stress coupling mechanism for intersections between hydraulic fractures and natural fractures //Journal of Petroleum Science and Engineering. – 2018. – Т. 171. – С. 37-47.
\bibitem{Paul2018}
Paul B. et al. 3D coupled HM–XFEM modeling with cohesive zone model and applications to non planar hydraulic fracture propagation and multiple hydraulic fractures interference //Computer Methods in Applied Mechanics and Engineering. – 2018. – Т. 342. – С. 321-353.
\bibitem{Duarte2019}
Shauer N., Duarte C. A. Improved algorithms for generalized finite element simulations of three‐dimensional hydraulic fracture propagation //International Journal for Numerical and Analytical Methods in Geomechanics. – 2019. – Т. 43. – №. 18. – С. 2707-2742.
\bibitem{Duarte2020_validation}
Mukhtar F. M., Alves P. D., Duarte C. A. Validation of a 3-D adaptive stable generalized/eXtended finite element method for mixed-mode brittle fracture propagation //International Journal of Fracture. – 2020. – Т. 225. – №. 2. – С. 129-152.
\bibitem{Duarte2020}
Shauer N., Duarte C. A. A generalized finite element method for three-dimensional hydraulic fracture propagation: Comparison with experiments //Engineering Fracture Mechanics. – 2020. – Т. 235. – С. 107098.
\bibitem{Roth2020_1}
Roth S. N., Léger P., Soulaïmani A. Strongly coupled XFEM formulation for non-planar three-dimensional simulation of hydraulic fracturing with emphasis on concrete dams //Computer Methods in Applied Mechanics and Engineering. – 2020. – Т. 363. – С. 112899.
\bibitem{Roth2020_2}
Roth S. N., Léger P., Soulaïmani A. Fully-coupled hydro-mechanical cracking using XFEM in 3D for application to complex flow in discontinuities including drainage system //Computer Methods in Applied Mechanics and Engineering. – 2020. – Т. 370. – С. 113282.
\bibitem{Shi2021}
??Shi F., Liu J. A fully coupled hydromechanical XFEM model for the simulation of 3D non-planar fluid-driven fracture propagation //Computers and Geotechnics. – 2021. – Т. 132. – С. 103971.

\bibitem{Williams1961}
Williams M. L. The bending stress distribution at the base of a stationary crack. – 1961.
\bibitem{Moes2012}
Geniaut S., Massin P., Moës N. A stable 3D contact formulation using X-FEM //European Journal of Computational Mechanics/Revue Européenne de Mécanique Numérique. – 2007. – Т. 16. – №. 2. – С. 259-275.
\bibitem{Moes2013}
Siavelis M. et al. Large sliding contact along branched discontinuities with X-FEM //Computational mechanics. – 2013. – Т. 52. – №. 1. – С. 201-219.
\bibitem{Moes2016}
Ferté G., Massin P., Moës N. 3D crack propagation with cohesive elements in the extended finite element method //Computer Methods in Applied Mechanics and Engineering. – 2016. – Т. 300. – С. 347-374.
\bibitem{Moes2017}
Paul B. et al. An integration technique for 3D curved cracks and branched discontinuities within the extended Finite Element Method //Finite Elements in Analysis and Design. – 2017. – Т. 123. – С. 19-50.
\bibitem{Belytschko1999}
Belytschko T., Black T. Elastic crack growth in finite elements with minimal remeshing //International journal for numerical methods in engineering. – 1999. – Т. 45. – №. 5. – С. 601-620.
\bibitem{Gupta2013}
Gupta V. et al. A stable and optimally convergent generalized FEM (SGFEM) for linear elastic fracture mechanics //Computer methods in applied mechanics and engineering. – 2013. – Т. 266. – С. 23-39.
\bibitem{Gupta2015_sgfem}
Gupta V. et al. Stable GFEM (SGFEM): Improved conditioning and accuracy of GFEM/XFEM for three-dimensional fracture mechanics //Computer methods in applied mechanics and engineering. – 2015. – Т. 289. – С. 355-386.
\bibitem{Asadpoure2007}
Asadpoure A., Mohammadi S. Developing new enrichment functions for crack simulation in orthotropic media by the extended finite element method //International Journal for Numerical Methods in Engineering. – 2007. – Т. 69. – №. 10. – С. 2150-2172.
\bibitem{Hattori2012}
Hattori G. et al. New anisotropic crack-tip enrichment functions for the extended finite element method //Computational Mechanics. – 2012. – Т. 50. – №. 5. – С. 591-601.
\bibitem{Feulvarch2020}
Feulvarch E., Lacroix R., Deschanels H. A 3D locking-free XFEM formulation for the von Mises elasto-plastic analysis of cracks //Computer Methods in Applied Mechanics and Engineering. – 2020. – Т. 361. – С. 112805.
\bibitem{Westergaard1939}
??Westergaard H. M. Bearing pressures and cracks //Trans AIME, J. Appl. Mech. – 1939. – Т. 6. – С. 49-53.
\bibitem{Minnebo2012}
Minnebo H. Three‐dimensional integration strategies of singular functions introduced by the XFEM in the LEFM //International Journal for Numerical Methods in Engineering. – 2012. – Т. 92. – №. 13. – С. 1117-1138.
\bibitem{Nagashima2020}
NAGASHIMA T. Three-dimensional crack analyses under thermal stress field by XFEM using only the Heaviside step function //Mechanical Engineering Journal. – 2020. – Т. 7. – №. 4. – С. 20-00098-20-00098.
\bibitem{Belytschko1988}
Belytschko T., Fish J., Engelmann B. E. A finite element with embedded localization zones //Computer methods in applied mechanics and engineering. – 1988. – Т. 70. – №. 1. – С. 59-89.
\bibitem{Nikolakopoulos2020}
Nikolakopoulos K., Crete J. P., Longère P. Volume averaging based integration method in the context of XFEM-cohesive zone model coupling //Mechanics Research Communications. – 2020. – Т. 104. – С. 103485.


\bibitem{Schollmann2002}
Schollmann M. et al. A new criterion for the prediction of crack development in multiaxially loaded structures //International Journal of Fracture. – 2002. – Т. 117. – №. 2. – С. 129-141.
\bibitem{Irwin1997}
Irwin G. R. Analysis of stresses and strains near the end of a crack traversing a plate. – 1997.
\bibitem{Lazarus2003}
Lazarus V. Brittle fracture and fatigue propagation paths of 3D plane cracks under uniform remote tensile loading //International journal of Fracture. – 2003. – Т. 122. – №. 1. – С. 23-46.
\bibitem{Gupta2017_SIF}
Gupta P., Duarte C. A., Dhankhar A. Accuracy and robustness of stress intensity factor extraction methods for the generalized/eXtended Finite Element Method //Engineering Fracture Mechanics. – 2017. – Т. 179. – С. 120-153.
\bibitem{Dirgantara2002}
Dirgantara T., Aliabadi M. H. Stress intensity factors for cracks in thin plates //Engineering fracture mechanics. – 2002. – Т. 69. – №. 13. – С. 1465-1486.


\bibitem{Pisarenko1981}
Писаренко Г. С., Можаровский Н. С. Уравнения и краевые задачи теории пластичности и ползучести. --- Киев : Наукова думка, 1981. --- 496 с.
\bibitem{SoloveychikRoyakPersova2007}
Соловейчик Ю. Г., Рояк М. Э., Персова М. Г. Метод конечных элементов для решения скалярных и векторных задач. --- Новосибирск : Изд-­во НГТУ, 2007. --- 895 с.
\bibitem{Zienkiewicz1975}
Зенкевич О. Метод конечных элементов в технике. --- М. : Мир, 1975. --- 542 с.
\bibitem{Frolov1995}
Александров А. В., Алфутов Н. А., Астанин В. В. и др. Энциклопедия ”Машиностроение”. Том I-­3. ”Динамика и прочность машин. Теория механизмов и машин”. В 2-­х книгах. Кн. 2 / Под ред. Фролов К. В. (гл. ред.). --- М. : Машиностроение, 1995. --- 624 с.
\bibitem{Khoei2014}
Khoei A. R. Extended finite element method: theory and applications. – John Wiley \& Sons, 2014.
	
\end{thebibliography}

%%% Local Variables:
%%% mode: latex
%%% TeX-master: "rpz"
%%% End:
